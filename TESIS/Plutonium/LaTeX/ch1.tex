%% intro.tex

\chapter{Introduction and summary}
%%%%%%%%%%%%%%%%%%%%%%%%%%%%%%%%%%%%%%%%%%%%%%%%

\subsubsection{Structure of this thesis}

\cite{art1}
This thesis falls naturally into four parts, which are relatively
independent:
\begin{itemize}
\item Study of dissipation rate in deformed chaotic billiards (Chapters~2,
3 and 4),
\item Improved numerical methods for quantization of billiards
(Chapters~\ref{ch:ipwdm} and \ref{ch:verg}),
\item Half-plane scattering
%theory
approach to mesoscopic conductance
(Chapter~\ref{ch:qpc}), and
\item Design of an atom waveguide using two-color evanescent light fields
(Chapter~\ref{ch:atom}).
\end{itemize}

The first two parts form the main body of the thesis, and they are both
devoted to the study of billiard systems (hard-walled cavities
enclosing a region of free space)
in which the classical motion is chaotic.
The quantum mechanics of such systems has become known as the field of
`quantum chaos'.
The first part probably contains the most significant new physical results;
this is reflected in the choice of thesis title.
The second part can be viewed merely as a description of numerical
quantum-mechanical calculations that play a
supporting role in the first part.
However, there will also turn out to be a surprising reciprocal connection,
namely that results from the first part will provide a much-needed
explanation for the success of a very efficient numerical technique
in the second part.
The intertwining of these two subject areas had turned out to be one of the
most beautiful surprises in this body of research.

The third and fourth parts form essentially separate projects, and can
therefore be read independently.
However they do share with the rest of the thesis the common theme of
wave mechanics: the third presents a new approach to
the transport of quasiparticle waves in mesoscopic systems,
and the fourth models confined electromagnetic waves to trap and guide atoms
(which themselves can be treated as coherent matter waves).

The goals and subject matter of the four parts are sufficiently different
to merit individual introductions and summaries, which now follow without
further ado.



% 1111111111111111111111111111111111111111111111111111111111111111111111111111
\subsubsection{Chapters 2,3 and 4: Dissipation rate and
deformations of chaotic billiards}


The dynamics of a particle inside a cavity
(billiard) in $d=2$ or 3 dimensions 
is a major theme in studies of classical and quantum chaos
\cite{ottbook,hellerleshouches,berryleshouches}.
Whereas the physics of time-independent chaotic systems 
is extensively explored, less is known 
about the physics when such a system is `driven' (time-dependent
chaotic Hamiltonian).
The main exceptions are the studies of 
the kicked rotator and related systems \cite{qkr}. 
However, the rotator (with no kicks) 
is a $d=1$ integrable system, whereas we 
are interested in chaotic ($d\ge2$) cavities.


Driven cavities have been of special interest since the 1970s in
studies of the so-called `one-body' dissipation rate in vibrating nuclei 
\cite{wall,koonincl,kooninqm,jarz92,jarz93}.
A renewed interest in this problem is anticipated
in the field of mesoscopic physics. Quantum dots \cite{been,dittrich}
can be regarded as small 2D cavities whose shape 
is controlled by electrical gates. Quasiparticle motion inside the dot
can have long coherence (dephasing) times,
and enable the semiclassical regime to be approached (many wavelengths
across the system).


In Chapter~\ref{ch:review} I give tutorial review of the
theory of dissipation in general driven
ergodic systems, which is quite a young field.
The Hamiltonian is controlled by a single parameter $x$, whose
time-dependence will be $x(t)=A\sin(\omega t)$ where $A$ is the amplitude  
and $\omega$ is the driving frequency.
In both the classical (Section~\ref{sec:classreview}) and quantum-mechanical
(Section~\ref{sec:qmreview}) pictures, dissipation is a result of
{\em stochastic energy spreading}.
Once this spreading is established, the pictures can be unified \cite{doronfrc}.
Irreversible growth of energy (heating) is then a result of biased diffusion
(a random walk) in energy.
I will confine myself to a regime where linear response theory (LRT) is valid.
In the quantum case this is known as the Kubo formalism, although the
language of the energy spreading picture appears different (I connect the
two pictures in Section~\ref{sec:resp}).
The heating rate is given by
