%% intro.tex

\chapter{El Lenguaje}

% intro %

Antes del diseño de cualquier compilador se deben definir los tres tipos de lenguajes que harán parte de él, pues la implementación está altamente ligada a ellos. Estos tipos de lenguajes son:

\begin{itemize}
	\item Lenguaje de entrada: Es el lenguaje en el que está escrito en código fuente.
	\item Lenguaje de salida: Es el lenguaje en el que estará escrito el código destino.
	\item Lenguaje(s) de implementación: Es ó son los lenguajes en los que está escrito el compilador, pueden ser varios dependiendo de la implementación del mismo.
\end{itemize}

Modificar cualquiera de los anteriores lenguajes repercute enormemente en el diseño y la complejidad del compilador, si bien todos los compiladores tienen etapas similares, realizar un cambio abrupto supone modificar la gramática, la semántica, las representaciones intermedias, optimizaciones y obviamente, la generación de código.
Es por tal razón, que antes de iniciar el proceso de diseño y codificación del compilador, es menester definir cada uno de los anteriores lenguajes. Puesto que no se llevará a cabo una implementación de un compilador completo, no se requiere la definición completa de cada lenguaje. El objetivo central de la investigación es estudiar el parsing\footnote{Parsing se refiere al análisis sintáctico.} partiendo desde dos modelos computacionales que permitirán observar el comportamiento de los mismos.


\section{Lenguajes utilizados}
Dado que la teoría descrita en los próximos capítulos parte del \emph{cálculo lambda} es necesario un lenguaje funcional, en esta caso \emph{Haskell} es el lenguaje seleccionado que permitirá implementar éste modelo de análisis sintáctico bastante poderoso y flexible. El principal objetivo de ésta investigación es observar la relación, propiedades e interacciones entre el universo funcional e imperativo, partiendo de la conjetura que son dos universos computacionales totalmente diferentes, pero aplicado a la teoría de compiladores, particularmente al problema del parsing y su elegante solución en el paradigma funcional. Por lo anterior, es necesario usar un lenguaje imperativo, flexible, veloz y con características que permitan desarrollar técnicas funcionales (condición necesaria explicada en el siguiente capítulo), por ésto, el lenguaje seleccionado es \emph{C++}.

	\subsection{C++}
	\emph{C++} es quizás el lenguaje de programación más eficiente jamás creado, y ha mantenido su popularidad debido a su constante evolución. El diseño de \emph{C++} está basado en la idea de proveer:	
	\begin{itemize}
		\item Proveer un uso de memoria eficiente y operaciones eficientes de bajo nivel.
		\item Mecanismos de abstracción asequibles y flexibles. \cite{Bjarne2013}
	\end{itemize}
	
	\emph{C++} provee demasiados mecanismos de abstracción sin perder el \emph{performance}. Además de ser un lenguaje \emph{multiparadigma}, da la facilidad de utilizar funciones que se acercan mucho al mundo funcional, pero nunca perdiendo el imperativismo. Las últimas versiones del lenguaje incluyen una serie de elementos y abstracciones adicionales que proveen una mayor expresividad funcional.
	
	Al igual que en el universo funcional, Bjarne expresa que una de las principales formas de completar tareas complejas es la composición de funciones. Es por esto, que \emph{C++} permite múltiples formas de abstracción basándose en las funciones. Desde funciones \emph{genéricas}, hasta punteros a funciones, lo cuál permite crear algoritmos con técnicas funcionales sin desprenderse del universo imperativo. 
	
	Es necesario definir las características que permitirán que \emph{C++} pueda implementar los \emph{Parser Combinators}, y de que forma se pueden representar aspectos de la programación funcional dentro de un lenguaje mayormente imperativo como lo es \emph{C++}.
	
	\subsubsection{Templates}
		Un \emph{template} es una clase o una función que se parametrizará con un conjunto de tipos o valores. Se utilizan \emph{templates} para representar conceptos que son mejor entendidos como algo muy general desde el cual se pueden generar tipos específicos y funciones con argumentos que se pueden especificar. \cite{Bjarne2013}
		
		De una forma casi directa se puede asociar el concepto de \emph{template} con el concepto de \emph{polimorfísmo}\footnote{Polimorfísmo entendido dentro del universo funcional, cabe anotar que el polimorfismo en el universo imperativo tiene un significado totalmente distinto.}, y viceversa, es decir, se puede afirmar que el \emph{polimorfísmo} funcional está asociado al concepto de \emph{template}. Obviamente la limitante de los \emph{templates} es que \emph{C++} al ser un lenguaje compilado y con comprobación de tipos estática, se debe definir en algún momento que tipo de dato reemplazará a la definición general del \emph{template}.
		
		\begin{lstlisting}[language=C++, caption=Función que suma dos variables de tipo T]
		template <typename T>
		T sum(T a, T b)
		{
			return a + b;
		}
		\end{lstlisting}
		
		Así se puede generar una función con tipos no especificados, cabe anotar como se dijo anteriormente, en el momento del uso hay que especificar su tipo.
	
	\subsubsection{Puntero a una función}
		Como un objeto, el código generado para un cuerpo de una función es puesto en memoria en algún lugar, entonces tiene una dirección. Se puede tener un puntero a una función como un puntero a un objeto. \cite{Bjarne2013}
		
		\begin{lstlisting}[language=C++, caption=Puntero a una función]
		void func(string s) { /* ... */ }
		
		void (*pf)(string);
		
		void f()
		{
			pf = &error;
			pf("POINTER TO FUNCTION!");
		}
		\end{lstlisting}
		
		De ésta manera se puede utilizar el puntero \emph{pf} que tiene la posición de memoria de la función \emph{func}. A simple vista puede parecer inútil, pero éstos punteros son la base para poder simular las \emph{funciones de primer orden} presentes en lenguajes de programación funcional, como lo és \emph{Haskell}.
		
		\begin{lstlisting}[language=C++, caption=Funciones de primer orden]
		void func(string s) { /* ... */ }
		
		void high_order_function(void (*pf)(string), string a)
		{
			pf(a);
		}
		\end{lstlisting}
		
		De ésta manera se pueden pasar funciones como argumentos. Para evitar escribir explícitamente el tipo de la función, se puede recurrir al uso de \emph{templates}.
		
		\begin{lstlisting}[language=C++, caption=Funciones de primer orden con templates]
		void func(string s) { /* ... */ }
		
		template<typename F>
		void high_order_function(F f, string a)
		{ 
			f(a);
		}
		\end{lstlisting}
		
	\subsubsection{Expresiones Lambda}
		\emph{C++11} tiene implementado una eficiente, fácil y elegante forma de utilizar las \emph{expresiones lambda} pertenecientes al universo funcional, las cuales no son otra cosa que funciones anónimas, en otras palabras, una expresión lambda no es más que una función sin nombre, es decir, no están muy separadas del universo imperativo. Utilizando \emph{lambdas} se puede evitar los punteros a funciones.
		
		Tal como lo expresa Stroustrup, una expresión lambda es una notación simplificada para definir y usar un objeto a una función anónima. En lugar de definir una clase con el operador (), luego crear el objeto y finalmente invocarlo, se puede usar un atajo. Esto es particularmente útil cuando se necesita pasar una operación como argumento a un algoritmo.\cite{Bjarne2013}
		
		Se puede observar que el concepto de \emph{expresión lambda} no es del todo ajeno al universo imperativo, pero en el modelo computacional no existe éste concepto.
		
		Con una sintaxis bastante funcional se utilizan las \emph{expresiones lambda} en \emph{C++}:
		
		\begin{lstlisting}[language=C++, caption=Lambda para elevar al cuadrado un número en C++]
			[](int x) -> int { return x * x; }
		\end{lstlisting}
	
	\subsubsection{Auto type}
		\emph{C++11} trae también consigo la detección automática de tipos, obviamente, como ya se a expresado, es necesario que éste tipo sea resuelto en tiempo de compilación, es decir, la detección de tipos no puede estar asociada a un efectos colateral, en otras palabras, si el chequeo de tipos no se puede hacer de forma estática, se debe utilizar un tipo explícito.
	
		\begin{lstlisting}[language=C++, caption=Auto type en C++]
			auto num = 2.8 * 3; //auto -> double
		\end{lstlisting}
		
		De la misma forma se puede evitar escribir todo el tipo del puntero a una función y también se pueden crear variables asociadas a una función lambda:
		
		\begin{lstlisting}[language=C++, caption=Auto type en C++]
		auto flambda = [](int x) -> int { return x * x; };
		\end{lstlisting}
				
	\subsection{Haskell}
		\emph{Haskell} es un lenguaje de programación funcional creado principalmente por \textsc{Simon Peyton Jones}, su nombre proviene de \textsc{Haskell B. Curry} quien fue uno de los pioneros en el cálculo lambda. La primera especificación de \emph{Haskell} fue en 1980. \emph{Haskell} es un lenguaje funcional puro, es decir, utiliza funciones sin efectos colaterales \cite{Thompson2011}.
		
		\emph{Haskell} es sin lugar a dudas el lenguaje funcional por excelencia.
		
		\subsubsection{Currying}
		Toda función en \emph{Haskell} es expresada de forma currificada, así pues, una función que recibe múltiples parámetros, será traducida como múltiples funciones que reciben cada una un único parámetro.
		
		\subsubsection{Polimorfismo}
		Una de las principales características de \emph{Haskell} es el polimorfismo. Un tipo que contiene uno o más tipos de variables es llamado \emph{polimórfico}(muchas formas).\cite{Hutton2007}
		
		Si se quisiera definir una función que recibe una lista de cualquier tipo y retorna el número de elementos se debe utilizar el polimorfismo.
		
		\begin{lstlisting}[language=Haskell, caption=Polimorfismo]
		size		::	[a] -> Int
		size 	[]	= 0
		size (x:xs) = 1 + size xs
		\end{lstlisting}
		
		\subsubsection{List comprehension}
		En matemáticas, la notación por comprensión puede utilizarse para construir nuevos conjuntos de otros existentes. Por ejemplo, $\{x^2 | x \in [1..5]\}$ produce el conjunto $[1,4,9,16,25]$. En \emph{Haskell} existe una notación similar: \cite{Hutton2007}
		\begin{lstlisting}[language=Haskell, caption=List comprehension]
		[x ^ 2 | x <- [1..5]]
		\end{lstlisting}



