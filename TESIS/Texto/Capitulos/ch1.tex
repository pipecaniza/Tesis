%% intro.tex

\chapter{El Lenguaje}

% intro %

Antes del diseño de cualquier compilador se deben definir los tres tipos de lenguajes que harán parte de él, pues la implementación está altamente ligada a ellos. Estos tipos de lenguajes son:

\begin{itemize}
	\item Lenguaje de entrada: Es el lenguaje en el que está escrito en código fuente.
	\item Lenguaje de salida: Es el lenguaje en el que estará escrito el código destino.
	\item Lenguaje(s) de implementación: Es ó son los lenguajes en los que está escrito el compilador, pueden ser varios dependiendo de la implementación del mismo.
\end{itemize}

Modificar cualquiera de los anteriores lenguajes repercute enormemente en el diseño y la complejidad del compilador, si bien todos los compiladores tienen etapas similares, realizar un cambio abrupto supone modificar la gramática, la semántica, las representaciones intermedias, optimizaciones y obviamente, la generación de código.
Es por tal razón, que antes de iniciar el proceso de diseño y codificación del compilador, es menester definir cada uno de los anteriores lenguajes.

\section{Lenguaje de entrada}
	\subsection{Objetivos del lenguaje}
	
	Si el lenguaje de entrada es muy complejo necesita un compilador de igual complejidad, por tal razón, el primer objetivo es definir un lenguaje sumamente sencillo.\\\\
	\noindent
	Como segundo objetivo, el lenguaje debe ser débilmente tipado, para facilitar su uso.\\\\
	\noindent
	El tercer objetivo , un lenguaje matemático, es decir, con expresividad matemática, que sea cómodo trabajar estructuras matemáticas dentro del mismo.\\\\
	\noindent
	El cuarto y último objetivo, un lenguaje basado en funciones, su bloque elemental es la definición de una función.
	
	\subsection{Definición del lenguaje}
	
	\subsection{Gramática}

\section{Lenguaje de Salida e Implementación}
	\subsection{C++}
	


\section{Lenguaje de Implementación}
	\subsection{Haskell}



