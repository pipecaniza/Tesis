%% Marco Teórico C1 %%

\chapter{Paradigmas}

``Las herramientas que utilizamos tienen una profunda (¡y retorcida!) influencia en nuestros hábitos de pensamiento, y, por lo tanto, en nuestra habilidad mental'', Edsger Dijkstra. \\

Los paradigmas de programación son modelos que guían la forma en que debemos pensar para resolver problemas utilizando herramientas computacionales. Cada paradigma tiene sus ventajas y desventajas de acuerdo al tipo de problema que se deseé resolver. En este capítulo revisaremos los dos principales modelos que soportan las corrientes  \emph{imperativa} y \emph{funcional}.

\section{Modelo computacional}



\subsection{Máquina de Turing}

\subsection{Cálculo Lambda}


\section{Lenguajes de programación}

\subsection{Imperativo}

\subsection{Declarativo}