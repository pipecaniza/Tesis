%% Marco Teórico C2 %%

\chapter{Teoría Funcional}

Llamamos programación funcional a un paradigma, el cual comenzó a principios de los sesenta, impulsado por la necesidades de los científicos de la inteligencia artificial, y que cubriera los procesos del cálculo simbólico, teoría de pruebas, pruebas de teoremas, en tal época ningún lenguaje imperativo daba una aproximación a estas necesidades.\\
La programación funcional trata los cálculos como funciones matemáticas, no hay la noción de posición de memoria, por tanto tampoco asignación, a través de la aplicación de los teoremas de recursividad se operan los bucles, también se distingue la notación frente a lenguajes imperativos clásicos. Hoy en día, algunos lenguajes de programación han tratados de implementar este paradigma.\\
La programación funcional se cimenta en el cálculo lambda desarrollado por Church en los treinta para dar una teoría paralela de funciones, este provee a la programación de sintaxis, semántica para definir función y permite definir primitivas de programación.\\
En la actualidad los lenguajes funcionales son Haskell, Scheme, Erlang, Rust, OCaml, el interés principal de la programación funcional es académico, aunque a veces se usa en la industria en ciertos desarrollos, por ejemplo con F#.
