%% Marco Teórico C2 %%

\chapter{Teoría Funcional}

Llamamos programación funcional a un paradigma de computación, el cual comenzó a principios de los sesenta, impulsado por las necesidades de científicos de la inteligencia artificial de comprender y programar los procesos del cálculo simbólico, teoría de pruebas y pruebas de teoremas ya en tal época ningún lenguaje imperativo daba una aproximación a estas necesidades.\\
La programación funcional trata los cálculos como funciones matemáticas, no existe la noción de posición de memoria, por tanto tampoco de asignación, a través de la aplicación de los teoremas de recursividad se operan los bucles, también se distingue la notación frente a lenguajes imperativos clásicos. Hoy en día, algunos lenguajes de programación han tratados de implementar este paradigma.\\
La programación funcional se cimenta en el cálculo lambda desarrollado por Church en los treinta para dar una teoría paralela de funciones, este provee a la programación de sintaxis, semántica para definir función y permite definir primitivas de programación.

\section{Lenguajes funcionales}

\subsection{Funciones}

Antes de que los ordenadores vieran la luz se habian concedido tecnicas para solucionar problemas. Por anto la terminologia referente a memoria no se usaba. En la matemática un concepto fundamental  es el de funciónlas funcione, las cuales actuan como un procesador con una entrada y dando como resultado una salida donde los resultados dependen de los parametros a traves de cada pasó, es por eso que se puede asociar una función, con la computación.\\

Podemos entonces decir que un ``programa'' consiste en la definicion de una o varias funciones. Donde el ordenador procesa los parametros en la funcion  entrega el resultado. Con esta aproximacion no tenemos que preocuparnos del orden de ejecucion, lo cual ya es mas potente que la aproximacion imperativa.
Al pasar el tiempo evidentemente esta tendencia fue perdiendo fuerza en el mercado pero nunca en la academia 

\subsection{Lenguajes}

A finales de los cincuenta John McCarthy creo el primer lenguaje de programación funcional Lisp. Posteriormente y por la necesidad de dar mas fortaleza al tipado se crearon otros, ML, Scheme, Hope. Cada investigador se enfocaba en crear el suyo propio, tratando de tomar lo mejor de cada uno, un grupo de cientificos de la computacion creo Haskell en los noventa. Posteriormente vendrian versiones de los que ya existian y algunos nuevos Clojure, Scala entre otros, algunos ya tomando fuerza en desarrollos empresariales, por ejemplo F\#.

\subsection{Teoria de Categorias y Programación Funcional}

Muchos conceptos de ka programación funcional tienen su origen en la teoria de categorias, por ello introducimos a partir de ahora esta teoria de manera formal como sustento axiomatico de nuestro trabajo.

\subsection{Teoria de Categorias}

\begin{defn}[Categoria]
Una categoría C consta de los siguientes datos:
\begin{itemize}
\item Conjunto $Ob(C)$ de objetos, $A, B, \dots$
\item Conjunto $Mor(C)$ de morfismos $f, g, \dots$
\item Una aplicación $dom : Mor(C) \to Ob(C)$ que a cada $f \in Mor(C)$ le
asigna el objeto $dom(f)$, llamado dominio de f.
\item Una aplicación $cod : Mor(C) \to Ob(C)$ que a cada $f \in Mor(C)$ le
asigna el objeto $cod(f)$, llamado codominio de f.
\item Una aplicación $I_A : Mor(C) \to Ob(C)$ que a cada $A \in Obj(C)$ le
asigna el morfismo $I_A$, llamado morfismo identidad de $A$.
\item Siendo $\displaystyle \prod_{Obj(C)} Mor(C)^2 := \left\lbrace (f, g) \in Mor(C \times C) | dom(f)=cod(g) \right\rbrace$, construimos una aplicacion llamada composicion.
\begin{align*}
  \displaystyle \circ: & \prod_{Obj(C)} Mor(C)^2  \to Mor(C) \\
					   & (f,g) 					  \mapsto f \circ g
\end{align*}
\end{itemize}
\end{defn}

\begin{note}
Si $A,B \in Ob(C)$ entonces 

\begin{itemize}
\item $Hom_{C}(A,B):= \{ f \in Mor_{C}(dom(f)=A \wedge cod(f)=B)) \}$
\item $f: A \to B$ significa que $f \in Hom_{C}(A,B)$
\item Si $f,g,h \in Mor(C)$ y $f:A \to B.g:B \to C.h:C \to A$ entonces los siguientes diagramas conmutan

	\xymatrix{
	A \ar[r]^{f} \ar[rd]_{g \circ f} &
	B \ar[d]^{g} \ar[rd]^{h \circ g} \\ 
	& C \ar[r]_{h} &
	D}	
	\xymatrix{
	A \ar[r]^{I_A} \ar[rd]_{f} &
	A \ar[d]^{f}  \\
	& B }
	\xymatrix{
	A \ar[r]^{f} \ar[rd]_{f} &
	B \ar[d]^{I_B}  \\
	& B }
	
\end{itemize}
\end{note}

\begin{defn}[Monoide]
Es una tupla $\langle M, *, \mft{a}, e, \mft{i} \rangle$ con los siguientes elementos
\begin{itemize}
\item $M$ Conjunto
\item $\mu$ Operador $* : M \times M \to M$
\item $\mathcal{a}$ Asociatividad $\forall_{m_1, m_2, m_3 \in M} ((m_1 * m_2) * m_3 = m_1 * (m_2 * m_3))$
\item $e$ Elemento Identidad [$\mft{i}$] $\forall_{m \in M} ( e * m = m = m * e)$
\end{itemize}
\end{defn}