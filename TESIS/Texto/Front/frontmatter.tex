%% frontmatter.tex
%%

\titulo{Parser combinator pseudo-funcional}
\autorA{Juan Felipe Cañizares Corrales}
\autorB{Lukas Restrepo Suarez}
\degreemonth{Julio} % month final submission occurs.
\degreeyear{2015}
\facultad{factultad de ingeniería}
\programa{ingeniería de sistemas}
\titulooptar{ingeniero de sistemas}
\asesorA{Daniel Cañizares Corrales}
\asesorB{Mario Zuluaga Tobón}

\fecha{03/11/2015}
\ciudad{Rionegro}

\maketitle
\copyrightpage
\entregapage




%We show how a systematic subtraction of the `special' components of a general
%deformation can be used to give an improved
%version of the `wall formula' estimate for $\mu(0)$.
%We believe this is the first study of $\omega$-dependent heating rate in
%billards, and the first consideration of the `special' nature of dilation.



% these are optional in the Jan 2000 Harvard thesis GSAS guide:
%\listoffigures
%\listoftables
%(Cut them for my personal thesis format).



% Dedicatoria %
\begin{dedication}
	\begin{quote}
		\hsp
		\em
		\raggedleft
		
		Dedicado a las List of Comprehension,\\
		sin ellas este trabajo\\
		no hubiera sido posible.
		
	\end{quote}
\end{dedication}

% Agradecimiento %
\begin{acknowledgments}
A los profesores Daniel Cañizares Corrales y Mario Zuluaga Tobón, quienes en su figura de asesores apoyaron enormemente este trabajo.\\
Al profesor Juan Francisco Cardona McCormick, quien nos brindó un gran apoyo y sus conocimientos, sin él este trabajo no hubiera sido posible.\\
Al profesor Erik Meijer, quien nos introdujo en el mundo funcional.\\
Al profesor Fabio Avellaneda Pachón, quien nos ayudo al inicio de este trabajo.\\
A Fabio Cañizares Cüeppers, quien ayudó a corregir la ortografía. La única persona que ha leído y leerá el trabajo completo dos veces.\\
A Diana Corrales Aponte, quien durante 9 meses no se cansó de escuchar la palabra: \emph{Parser Combinators}.\\
Al procesador de nuestra máquina, quien soportó los innumerables errores de programación que cometimos durante este trabajo; siempre incansable nos permitió usar al menos uno de sus núcleos.\\



Agradecemos a nuestros seres queridos por su apoyo.\\
\vspace*{5cm}
\begin{quote}
	\hsp
	\em
	\raggedleft
	
	``Usted, estoy seguro, estará de acuerdo conmigo... en que si la página 534 está en el capítulo dos, la longitud del primero ha sido realmente intolerable."\\ Sherlock Holmes
	
\end{quote}

\end{acknowledgments}

% Tabla de contenido %
\addcontentsline{toc}{section}{Tabla de Contenido}
\tableofcontents

\newpage

% Abstract - Resumen %
\begin{abstract}
\vspace{1cm}
En esta investigación se definen todos los elementos de la teoría de computación, de categorías y de compiladores necesarios para la implementación de Parser Combinators Pseudo-Funcionales, estos son llamado Pseudo ya que en el proceso de la elaboración del trabajo hemos podido conjeturar sobre la real aproximación de la programación puramente funcional sobre arquitecturas Von Neumann, como resultado y prueba de este proceso se logró la construcción de una librería que contiene un parser combinator construido en Haskell y un parser combinator en C++, basados en una teoría puramente funcional lo cual nos permite dar una representación tangible de nuestra teoría.
\end{abstract}

\startarabicpagination

%%% end

